\newcommand{\doctitle}{Stratified singular spaces and poincarè-duality by Markus Banagl - 12.04.11}
\documentclass[12pt, a4paper]{scrartcl}
\usepackage[utf8]{inputenc}
\usepackage[T1]{fontenc}
\usepackage[english]{babel}

\usepackage{graphicx}
\usepackage{url}
\usepackage{listings}
\usepackage{palatino,avant}
\usepackage{amsmath,amssymb,amsthm,mathrsfs,stmaryrd,mathabx,yfonts} % mathabx
\usepackage{pb-diagram}
\usepackage{cancel}
\usepackage{txfonts}
\usepackage{graphics}
\usepackage[pdftex,bookmarks=true,bookmarksnumbered=true,bookmarksopen=true,colorlinks=true,filecolor=black,
                linkcolor=red,urlcolor=blue,plainpages=false,pdfpagelabels,citecolor=black,
                pdftitle={\doctitle},setpagesize=false,pdfauthor={Axel Wagner}]{hyperref}

\newcommand{\makelicense}{Dieser Mitschrieb steht unter der freien \texttt{CC-BY-SA-DE 3.0} Lizenz.
\begin{center}\includegraphics[height=31pt,width=88pt]{license.pdf}\end{center}
Für weitere Informationen besuchen Sie
\begin{center}\url{http://creativecommons.org/licenses/by-sa/3.0/deed.de}\end{center}}

\newcommand{\ra}{\rightarrow}
\newcommand{\Ra}{\Rightarrow}
\newcommand{\La}{\Leftarrow}
\newcommand{\LRa}{\Leftrightarrow}
\newcommand{\bi}{\leftrightarrow}
\renewcommand{\c}{\colon}
\newcommand{\ce}{\coloneqq}
\newcommand{\ec}{\eqqcolon}
\newcommand{\mb}{\mathbb}
\newcommand{\mf}{\mathfrak}
\newcommand{\mc}{\mathcal}
\newcommand{\ms}{\mathscr}
\newcommand{\K}{\mb{K}}
\newcommand{\R}{\mb{R}}
\newcommand{\Q}{\mb{Q}}
\newcommand{\N}{\mb{N}}
\newcommand{\C}{\mb{C}}
\newcommand{\Z}{\mb{Z}}
\renewcommand{\H}{\mb{H}}
\newcommand{\T}{\mathcal{T}}
\newcommand{\A}{\mathcal{A}}
\newcommand{\D}{\mathcal{D}}
\newcommand{\GL}{\text{GL}}
\newcommand{\B}{\mathcal{B}}
\renewcommand{\S}{\mathcal{S}}
\renewcommand{\a}{\mathfrak{a}}
\renewcommand{\b}{\mathfrak{b}}
\newcommand{\g}{\mathfrak{g}}
\newcommand{\h}{\mathfrak{h}}
\newcommand{\id}{\mathrm{id}}
\newcommand{\diam}{\mathrm{diam}}
\newcommand{\supp}{\mathrm{supp}}
\newcommand{\rk}{\mathrm{rk}}
\newcommand{\Alt}{\mathrm{Alt}}
\renewcommand{\deg}{\mathrm{deg}}
\newcommand{\Hom}{\text{Hom}}
\newcommand{\del}{\partial}
\renewcommand{\d}{\,\mathrm{d}}
\renewcommand{\D}{\mathrm{D}}
\newcommand{\Ta}{\mathrm{T}}
\newcommand{\offen}{\underset{\mathrm{offen}}{\subset}}
\newcommand{\abgeschl}{\underset{\mathrm{abgeschl.}}{\subset}}
\newcommand{\lsup}[2]{\ensuremath{{}^{#2}\smallskip{#1}}}
\newcommand{\keime}{\mathrm{Keime}}
\newcommand{\ind}{\mathrm{Index}}
\newcommand{\im}{\mathrm{Im}\,}
\newcommand{\interior}{\mathrm{\int}\,}
\newcommand{\rot}{\mathrm{rot}}
\newcommand{\sgn}{\mathrm{sgn}}
\newcommand{\mathhead}[1]{\texorpdfstring{$#1$}{#1}}
\newcommand{\UVR}{\stackrel{\text{UVR}}{\subset}}
\newcommand{\Span}{\mathrm{span}}
\newcommand{\dist}{\mathrm{dist}}
\renewcommand{\L}{\mathscr{L}}
\newcommand{\PD}{Poincaré-duality}
\newcommand{\quot}[2]{#1 / #2}

\newtheorem{definition}{Definition}[section]
\newtheorem{lemma}[definition]{Lemma}
\newtheorem{satz}[definition]{Theorem}
\newtheorem{korollar}[definition]{Corollary}
\newtheorem{proposition}[definition]{Proposition}

% No indention at new paragraphs
\setlength{\parindent}{0pt}

\begin{document}

\makelicense


\section*{Motivation}

\begin{satz}[\PD\ for manifolds]
    Let $M^n$ be a closed, oriented manifold.
    \[ [M] \in H_n(M) \]
    \[ -\cap [M]\c H^i(M) \stackrel{\cong}{\to} H_{n-i}(M) \]
    In particular
    \[ b_i(M) \ce \rk H_i(M) = \rk H_{n-i}(M) \]
\end{satz}

Singular spaces do not posses \PD.

\textbf{Example:} ”Pinched Torus“ % Image
\[ H_0(X) \cong \Z \]
\[ H_2(X) \cong \Z \]
Problem: $H_1(X) = \Z$. There is no intesersection product, so \PD\ does not hold.

\textbf{Example:} $X^3 = \Sigma(T^2)$, the suspension of the Torus. % Image
\[ b_0 = 1 = b_3 \]
\[ b_1 = 0 \]
\[ b_2 = 2 \]
Since $b_1 \neq b_2$, \PD\ does not hold.

\section*{Stratified spaces}

A \textbf{stratification} of a singular space $X$ will turn out to be a Decomposition of $X$ into a disjoint union of $S_i$, such that the $S_i$ are manifolds and points in $S_i$ are ”equaly singular“.

\subsection*{Topological construction}

\begin{itemize}
    \item A Cross
    \item The pinched torus % Image ”Topological constructions 1“
    \item $X^3 = \Sigma T^2 = \{N,S\} \cup \underbrace{\left(X^3 \setminus \{N,S\}\right)}_{\cong (0,1) \times T^2}$
    \item $X^2 = T^2 \cup T^2 = S_1 \cup S_2$ % Image ”Rotated figure 8“
    \item $X^4 = S^1 \times \Sigma T^2$
    \item $X^4 = \Sigma\Sigma T^2$ % Image ”Highly suspended torus“
\end{itemize}

\subsection*{Smooth groupactions}

Let $M^n$ be a closed smooth manifold and $G$ a compact Lie group acting smoothly on $M$.

\textbf{Example:} $G = S^1, M = S^2$ % Image ”Rotating orb“

$\rightsquigarrow$ Orbit Space $\quot{M}{G}$

The orbit space will usually be singular.

We will stratify $\quot{M}{G}$. Let $H$ be a closed Subgroup $H < G$.
\[ M_{(H)} \ce \{ x \in M \mid G_x \sim H \} \]
Where $\sim$ is conjugate equivalence, $G_x$ is the isotropy group at $x$ and $(H)$ is the conjugate class of $H$.

$M_{(H)}$ is a union of orbits:
\[ G_{g\cdot x} = gG_xGg^{-1} \sim G_x \sim H \]
\[ \Ra G_{g\cdot x} \sim H \]
\begin{proposition}
    Assume $\quot{M}{G}$ is connected. Then there exists a unique conjugate class $(H_0)$, such that $M_{(H_0)}$ is open and dense in $M$.
\end{proposition}

\textbf{Example:} In the rotating sphere example, $(H_0) = (e_0)$, because $M_{(e)} = S^2 \setminus \{N,S\}$.

\begin{itemize}
    \item $M_{(H)}$ is a smoothly embedded submanifold of $M$, called the orbit bundle of $(H)$.
    \[\begin{diagram}
        \node{\quot{G}{H}}
            \arrow{e}
        \node{M_{(H)}}
            \arrow{s}{\pi} \\
        \node{}
        \node{\quot{M_{(H)}}{G}}
    \end{diagram}\]
    \item As $G$ is compact, there are only finitely many isotropy types.
    \[ M = \bigcup_{i=1}^n M_{(H_i)} \]
    \[ \quot{M}{G} = \bigcup_{i=1}^n \underbrace{\quot{M_{(H_i)}}{G}}_{=S_{(H)}} \]
\end{document}

\textbf{Example:} In the rotating sphere example, $\quot{M}{G} = \quot{M_{(e)}}{G} \cup \quot{\{ N, S\}}{G}$.

\subsection*{Link}

% Image ”links“

Suppose $(H_0) = (e)$, there are only 2 isotropy types, $(e)$ and $(H)$, $H \neq e$, and $G$ acts transitively on $M_{(H)}$ (so there is some $x\in M_{(H)}$, such that $M_{(H)} = Gx$).

% Image ”Slice“
\textbf{Question:} What is the link of the singular point in $\quot{M}{G}$.
\[ V_x \ce \frac{T_xM}{T_x(Gx)} \]
$V_x$ is called a ”slice“.

Let $h \in G_x \sim H$.
\[ h_* \c T_x(Gx) \to T_{hx}(Gx) = T_x(Gx) \]
We obtain a represantation
\[ H \sim G_x \to \GL(V_x) \]
called the \textbf{slice-representation}.

We can therefore form
\[ G \times_{G_x} V_x \ce \quot{G\times V_x}{G_x} \]
which is a vector bundle over $Gx$.

\begin{satz}[Slice theorem]
    There exists an open neighbourhood $N_x$ of the orbit $Gx$ and an equivariant diffeomorphism
    \[ N_x \cong G \times_{G_x} V_x \]
\end{satz}
